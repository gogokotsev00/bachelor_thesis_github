\chapter{Introduction}
%labels will help you to reference to certain images, tables, chapters, section, and so on...
\label{introduction}
This chapter presents an explanation about what has served as motivation for this thesis. What else is provided are the approach for solving the issue as well as the aims of the work. At the end of the chapter, the structure is defined.

%###################################################################################
%###################### Motivation          ########################################
%###################################################################################
\section{Motivation}
In recent times, the amount of interest in a cooperative environment for autonomous vehicles and infrastructure has increased and there are many researches on this topic \cite{ cvis_article_two, cvis_article_one}. The main goal of it is to increase safety, decrease accidents and improve city traffic. In order to achieve this it must be guaranteed that the self-steering vehicles are aware of their surroundings during the whole ride. Today's the cars with the highest level of autonomy rely on numerous sensors like radars, Lidars and cameras to be able to perceive the other participants in the traffic. However, on-board sensors have their drawbacks like limited visibility and decreased field of perception, therefore the use of stationary sensors in the city infrastructure can be considered beneficial for improvement in the provided information. Placement of cameras in the road area significantly expands the vehicle's range of view and contributes to safer and congestion-free roads. Furthermore, cameras are cheaper and much easier to  maintain than the other sensors. Another advantage is that they are capable of seeing the world like a human being would do, which is of utmost importance by object detection.  

DELETEME: This section is very important since it argues why it is necessary to take care of the problem you are addressing in your work. One way to do this is coming from a very broad view on the problem to a very detailed one. This can be done by establishing a chain of statements that refer to each other until you reach your particular problem. Doing this, you really need to take care for citing every statement. 

DELETEME: Example for a chain: Mobile communication gets increasingly popular in the world (CITE sales on mobile communication infrastructure, mobile phones, or increasing number of mobile phones contracts). $\rightarrow$ Especially smartphones, which represent the next generation cellular phone (CITE), get more and more used for communicating not only with other people but also for connecting to the Internet for using various services (CITE). $\rightarrow$ Smartphone are comprehensive cellular phones that provide additional functionality due to their increased connection and processing capabilities (CITE). $\rightarrow$ Most smartphones offer an online application store for adding software to the devices which helps the users to customize their devices according to their needs, e.g. Android Market\footnote{\url{http://market.android.com}, visited on 05/08/2011}. $\rightarrow$ One problem about installing third-party software is that not all software try to help the user; $\rightarrow$ software with malicious intentions, so-called malicious software (malware), can be a severe threat to smartphone users. Some malware delete files (EXAMPLE + CITE or footnote with URL) or even destroy devices (EXAMPLE + CITE or footnote with URL). $\rightarrow$ More and more smartphone malware appeared in the last years (CITE). $\rightarrow$ Signature-based approaches work efficiently on known malware (CITE) but face serious drawbacks regarding unknown malware. $\rightarrow$ Oberheide et al.~\cite{oberheide:2008:cloudav} state that virus engines need an average time of 48 days until their databases get updated to be able to detect a certain unknown malware. $\rightarrow$ This in turn means that smartphone users stay unprotected for this time, which can be seen as a severe threat. $\rightarrow$ Therefore, approaches are needed that are capable of detecting unknown malware for protecting the users against such threats.
DELETEME: This example showed how one could argue that alternative approaches for malware detection is required. The length of the motivation depends on the topics handled and can of course be longer. The principle I am describing is also shown on Figure~\ref{fig:writing}

\begin{figure}
\centering
\includegraphics[width=0.9\textwidth]{template/writing}
\caption[Information Generality]{This images illustrates how generality of information could be handled in a thesis. In your motivation you should start from a very broad view on the topic. Then you should get more precise with every statement until you reach the actual problem you are addressing. You should do vice-versa in your conclusion, starting with the problem that you addressed and getting broader until you can write about the meaning of your results to the (IT-)world.\label{fig:writing}}
\end{figure}


%###################################################################################
%###################### Approach and Goals  ########################################
%###################################################################################
\section{Approach and Goals}
DELETEME: In this section, you should cleary describe your approach that you are following in order to solve the underlaying problem of your thesis. Additionally, you should clearly state the goals of your work. This will not only help you supervizor to understand what you are doing, it will also help you to be sure on which topic you should evaluate.


%###################################################################################
%###################### Structure of the Thesis ####################################
%###################################################################################
\section{Structure of the Thesis}
DELETEME: This section does not require eloquent writing. It is just a presentation of what you will handle in each chapter starting with Chapter~\ref{background}.

DELETEME: Example: This thesis is structured as follows. In Chapter~\ref{background}, we discuss essential background related to the thesis topic. (SOME MORE SENTENCES). Chapter~\ref{mainone} represents a detailed analysis of the problem that will be addressed. In particular, (SOME MORE SENTENCES). In Chapter~\ref{maintwo}, our solution is presented. This solution covers ... (SOME MORE SENTENCES). Chapter~\ref{evaluation} evaluates our solution basing on our specified goals. (SOME MORE SENTENCES). In Chapter~\ref{conclusion}, we conclude. Chapter~\ref{appendices} gives additional related information on the topic of this thesis.

This thesis is structured as follows: 
\begin{itemize}
    \item Chapter 2 talks about scientific information regarding camera, etc.
    \item Chapter 3 reveals some state-of-the-art methods for sensor placement, etc.
    \item In chapter 4 the problem that this thesis is concerned with and an approach for solving it are presented.
    \item Chapter 5 introduces the results of the conducted experiments and how they were evaluated.
    \item In the 6. Chapter we summarize the outcome of our experiments and discuss the possible tasks for a future work on this issue.
\end{itemize} 
