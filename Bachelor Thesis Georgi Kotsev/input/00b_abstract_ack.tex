%#############################################################
%###################### Statement ############################
%#############################################################
\chapter*{Erkl{\"a}rung der Urheberschaft}
%this one needs to be signed for submission
Ich erkläre hiermit an Eides statt, dass ich die vorliegende Arbeit ohne Hilfe Dritter und ohne Benutzung anderer als der angegebenen Hilfsmittel angefertigt habe; die aus fremden Quellen direkt oder indirekt übernommenen Gedanken sind als solche kenntlich gemacht. Die Arbeit wurde bisher in gleicher oder ähnlicher Form in keiner anderen Prüfungsbehörde vorgelegt und auch noch nicht veröffentlicht.


\vspace{4cm}

Ort, Datum \hfill Unterschrift

%#############################################################
%###################### Abstract  ############################
%#############################################################
\newpage
\chapter*{Abstract}
DELETEME: An abstract is a teaser for your work. You try to convince a reader that it is worth reading your work. Normally, it makes sense to structure you abstract in this way: 
\begin{itemize}
\item one paragraph on the motivation to your topic
\item one paragraph on what approach you have chosen
\item and one paragraph on your results which may be presented in comparison to other approaches that try to solve the same or a similar problem.
\end{itemize}
Abstract should not exceed one page (aubrey's opinion)

%#############################################################
%###################### German Abstract ######################
%#############################################################
\newpage
\chapter*{Zusammenfassung}
DELETEME: translate to German to English or vice-versa.

%#############################################################
%###################### Acknowledgements #####################
%#############################################################
\newpage
\chapter*{Acknowledgements}
DELETEME: Thank you for the pranks
