%#############################################################
%###################### Statement ############################
%#############################################################
\chapter*{Erkl{\"a}rung der Urheberschaft}
%this one needs to be signed for submission
Ich erkläre hiermit an Eides statt, dass ich die vorliegende Arbeit ohne Hilfe Dritter und ohne Benutzung anderer als der angegebenen Hilfsmittel angefertigt habe; die aus fremden Quellen direkt oder indirekt übernommenen Gedanken sind als solche kenntlich gemacht. Die Arbeit wurde bisher in gleicher oder ähnlicher Form in keiner anderen Prüfungsbehörde vorgelegt und auch noch nicht veröffentlicht.


\vspace{4cm}

Ort, Datum:\hspace{0.25cm} Berlin, den 21.12.2022 \hfill Unterschrift: \hspace{1cm}

%#############################################################
%###################### Abstract  ############################
%#############################################################
\newpage
\chapter*{Abstract}
% DELETEME: An abstract is a teaser for your work. You try to convince a reader that it is worth reading your work. Normally, it makes sense to structure you abstract in this way:

Connected and Automated Mobility (CAM) is a widely researched area, where the goal is autonomous autos to become the main mean of transportation. In order to perceive the surrounding environment, autonomous vehicles rely on combination of sensors like cameras, LiDARs or radars. However, onboard sensors have limitations in their range of view, especially when larger vehicles cause occlusion and obstruct their sight to the other traffic participants. Therefore, it is of utmost importance to install roadside sensors like cameras in the infrastructure, which are going to aid vehicles in path prediction, accidents avoidance, etc. For the purpose of efficient performance, cameras should be placed in such positions that allow them to have an unobstructed view over the region of interest and detect fewer occlusions. This thesis proposes an approach for evaluating camera placement options and picking the best among the evaluated candidates. It uses the 3D simulator CARLA, where various simulations with two vehicles in the roles of 'occluder' and 'target' are executed. The main objective is to examine how height of sensor's location and size of vehicles impact the measured occlusion. During the experiments, the camera is placed in 8 manually selected positions are tested at different height levels, after which the maximal occlusion degree values for each spawn location are displayed using heatmaps. Apart from the visual representation of the results, we introduce two metrics M1 and M2, which compare the cameras' performance by two factors - the quantity of detected occlusions and the number of occlusions exceeding a certain threshold. In this way, we are able to provide two positions, which deliver promising output, and give the opportunity to extend our strategy for future studies by experimenting with other vehicles and reproducing real-world scenarios.       


%#############################################################
%###################### German Abstract ######################
%#############################################################
\newpage
\chapter*{Zusammenfassung}
% DELETEME: translate to German to English or vice-versa.

Connected and Automated Mobility (CAM) ist ein weithin erforschter Bereich, in dem autonome Autos zum Haupttransportmittel werden sollen. Um die Umgebung wahrzunehmen, verlassen sich autonome Fahrzeuge auf eine Kombination von Sensoren wie Kameras, LiDARs oder Radare. Die Sichtweite der Onboard-Sensoren ist jedoch begrenzt, insbesondere wenn größere Fahrzeuge die Sicht auf die anderen Verkehrsteilnehmer verdecken. Daher ist es äußerst wichtig, straßenseitige Sensoren wie Kameras in die Infrastruktur einzubauen, die den Fahrzeugen bei der Wegvorhersage, der Unfallvermeidung usw. helfen werden. Um eine effiziente Leistung zu erzielen, sollten die Kameras so platziert werden, dass sie einen ungehinderten Blick auf den interessierenden Bereich haben und weniger Verdeckungen erkennen. In dieser Arbeit wird ein Ansatz vorgeschlagen, um die Optionen für die Kameraplatzierung zu bewerten und die besten Platzierungen daraus zu finden. Sie verwendet den 3D-Simulator CARLA, in dem verschiedene Simulationen mit zwei Fahrzeugen in den Rollen 'Verdecker' und 'Ziel' durchgeführt werden. Das Hauptziel besteht darin, zu untersuchen, wie sich die Höhe des Sensorstandorts und die Größe der Fahrzeuge auf die gemessene Verdeckung auswirken. Bei den Experimenten wird die Kamera an 8 manuell ausgewählten Positionen in verschiedenen Höhen getestet. Anschließend werden die maximalen Werte des Verdeckungsgrads für jede Position mit Hilfe von Heatmaps dargestellt. Neben der visuellen Darstellung der Ergebnisse führen wir zwei Metriken M1 und M2 ein, die die Leistung der Kameras anhand von zwei Faktoren vergleichen - die Anzahl der erkannten Verdeckungen und die Anzahl der Verdeckungen, die einen bestimmten Schwellenwert überschreiten. Auf diese Weise sind wir in der Lage, zwei Positionen anzubieten, die vielversprechende Ergebnisse liefern und die Möglichkeit bieten, unsere Strategie für zukünftige Studien zu erweitern, indem wir mit anderen Fahrzeugen experimentieren und reale Szenarien reproduzieren.

%#############################################################
%###################### Acknowledgements #####################
%#############################################################
\newpage
\chapter*{Acknowledgements}
For the constant support during the implementation of the code and the writing of this thesis, whenever problems occured, I want to thank my adviser Dipl.-Inf. Martin Berger. Additionally, I thank the authors of this template, for the endless efforts to help me and enhance my work.
